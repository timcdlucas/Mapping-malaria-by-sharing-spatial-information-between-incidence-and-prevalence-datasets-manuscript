Which journal? Comp bio was original plan? go for that.

\begin{enumerate}
\def\labelenumi{\arabic{enumi}.}
\tightlist
\item
  Use Madagascar + Indonesia as case studies.
\item
  do cross validation on both pr and polygon. no good way to combine so
  just have to report both in every case
\item
  can't see good way to compare stacked generalisation with linear
  polygon so have to do all linear.
\item
  if this paper falls out easily can expand to ml and polygon as
  covariates. or project for ra?
\item
  probably want random cv and spatial cv?
\item
  might need more countries but do first draft like this.
\item
  spatial only. but need to choose year/year range. year is difficult as
  pr and polygons don't match.
\item
  compare Pearson and spearman. maybe Mae as well.
\item
  how to handle spatial priors? try 3 values and pick best CV in each
  case.
\item
  gives country X model X CV X metric X prevalence vs incidence.
\end{enumerate}

figure 1. data and cross validation. columns county. rows: data values,
CV group 1, CV group 2.

figure 2. predicted prevalence and incidence maps. but how with CV? all
data X model?

figure 3. accuracy summary. Dodge by model. facet by prevelance
incidence and county. symbol metric. colour by cv type. points for
individual CV, line for mean? but how to distinguish CV type line?

figure 4 predicted Vs observed. facet by country and prev inc. colour by
cv?

\subsection{intro}\label{intro}

malaria is bad but decreasing moving to strategy of elimination country
by country elimination requires maps of low prevalence areas but maps in
low prevalence areas are difficult sometimes little data traditional
prevalence mapping won't work we need new sources of data.

however surveillance data is sometimes high quality in these areas pixel
level maps from aggregates data is difficult (sturrock, inla, Leon,
others) not much information for covariates sturrock used two.
undeveloped area of statistics but see Leon.

Ideal situation is to combine data benefits of both but data are on
different scales Ewan says there's a relationship.

here we compared 3 models, point only, polygon only and joint models we
use Madagascar and Indonesia as case studies as they have good
surveillance data and good pr data. we find that\ldots{}

\subsection{methods}\label{methods}

malaria data prevalence data prevalence standardisation incidence data
geography counts to incidence rate

covariate data

full model full hierarchical equations linear predictor link to
prevalence prevalence to incidence multilevel

\subsubsection{Malaria data}\label{malaria-data}

We used two data sources that reflect malaria burden; point prevalence
surveys and polygon-level, aggregated incidence data. We selected
Madagascar and Indonesia as case examples as they have both good
surveillance data and good country wide surveys from approximately the
same time. Given the available data we chose to focus our analysis on
2012-2015 for Indonesia and 2013-2014 for Madagascar.

\subsubsection{Prevalence survey data}\label{prevalence-survey-data}

The prevalence survey data were extracted from the MAP prevalence survey
database {[}@{]}. As the prevalence surveys cover different are ranges
they were standardisePars using the model from {[}@{]}.

\subsubsection{Polygon incidence data}\label{polygon-incidence-data}

\subsubsection{Population data}\label{population-data}

\subsubsection{Covariate data}\label{covariate-data}

\subsubsection{The model}\label{the-model}

We compared three models 1) a full model with prevalence surveys and
aggregated incidence data 2) the submodel with only prevalence data and
3) the submodel with only aggregated incidence data.

The full, joint likelihood model is described as follows. Values at the
aggregate, polygon level are given the subscript \(a\) while pixel or
point level below are indexed with \(b\). The polygon case count data,
\(y_j\) is given a Poisson likelihood

\[y_a \sim \operatorname{Pois}(i_a\mathrm{pop_a})\]

where \(i_a\) is the estimated polygon incidence rate and
\(\mathrm{pop_a}\) is the observed polygon population-at-risk.

The point-level prevalence data, \(z_b\), is given a binomial likelihood

\[z_b \sim \operatorname{Binom}(p_b, n_b) \]

where \(p_b\) is the estimated prevalence and \(n_b\) is the observed
survey sample size.

The two quantities are linked to each other and to the predictor
variables by the following system of equations.

\[i_a = \frac{ \sum(i_b \mathrm{pop}_b)}{\sum  \mathrm{pop}_b} \]

\[i_b = \mathrm{p2i}(p_b)\]

where \(\mathrm{p2i}\) is a from a model that was published previously.
{[}@cameron2015defining{]} After fitting, this model defines a function
\[\mathrm{p2i}: f\left(P\right) = 2.616P - 3.596P^2 + 1.594P^3\].

The linear predictor of the model, \(\eta_b\), is related to prevalence
by a typical logit link function.

\[p_b = \operatorname{logit}^{-1}(\eta_b)\]

The linear predictor is composed of an intercept, covariates and a
spatial, Gaussian random field.

\[\eta_b = \beta_0 + \beta X  + u(s, \tau, \kappa)\]

The spatial effect \(u(s, \tau, \kappa)\) has a Mat'ern covariance
function and two hyper parameters. blah.

Finally, we complete the model by setting priors on the parameters
\(\beta_0, \beta, \tau\) and \(\kappa\).

We set

\[\log(\tau)\sim \operatorname{ Norm}(2, 6)\]

which corresponds to a mean nominal range \(r = \frac{\tau}{\sqrt{2}}\)
of one fifth of the range of the study area.

We set \[\log(\kappa)\sim \operatorname{ Norm}(2, 6)\]

for some reason. Finally, we set regularising priors on the regression
coefficients

\[\beta_i \sim \operatorname{ Norm}(0, \sigma)\]

where we try the values for \(\sigma\), 0.2, 1 and 2.

Given this setup, we get predictions of prevalence and incidence
simultaneously whether we used both data types or just one.

\subsubsection{experiments}\label{experiments}

To compare the three models we used two cross validation schemes. In the
first, the combined data set of prevalence and incidence data was split
into five cross-validation folds stratified by data type. This reflects
real world data where we sometimes have incidence data but no prevalence
data and visa versa. In the second validation scheme the incidence data
was split into five spatial cross-validation folds (see
Figure\textasciitilde{}\ref{fig:data}b). Then, any prevalence data
within the hold out incidence polygons was also withheld. This scheme is
testing the models' ability to predict into new areas with no
information from the spatial random field.

In both cases we examined performance metrics for both the withheld
prevalence data and the withheld incidence data. As there is no good way
of combining predictive error from both types of data into one
performance metric, we considered performance separately throughout. Our
main performance metric was Pearson's correlation. However, we also
considered Spearman's correlation. If the prevalence-incidence
relationship is poorly estimated, this will strongly affect the
Pearson's correlation while the Spearman's correlation will be
relatively robust to this source of poor model performance.

\subsection{results}\label{results}

\subsection{discussion}\label{discussion}

what did and didn't work application to global maps further ml work
potential way to refine prevalence incidence relationship.
