% Template for PLoS
% Version 3.5 March 2018
%
% % % % % % % % % % % % % % % % % % % % % %
%
% -- IMPORTANT NOTE
%
% This template contains comments intended 
% to minimize problems and delays during our production 
% process. Please follow the template instructions
% whenever possible.
%
% % % % % % % % % % % % % % % % % % % % % % % 
%
% Once your paper is accepted for publication, 
% PLEASE REMOVE ALL TRACKED CHANGES in this file 
% and leave only the final text of your manuscript. 
% PLOS recommends the use of latexdiff to track changes during review, as this will help to maintain a clean tex file.
% Visit https://www.ctan.org/pkg/latexdiff?lang=en for info or contact us at latex@plos.org.
%
%
% There are no restrictions on package use within the LaTeX files except that 
% no packages listed in the template may be deleted.
%
% Please do not include colors or graphics in the text.
%
% The manuscript LaTeX source should be contained within a single file (do not use \input, \externaldocument, or similar commands).
%
% % % % % % % % % % % % % % % % % % % % % % %
%
% -- FIGURES AND TABLES
%
% Please include tables/figure captions directly after the paragraph where they are first cited in the text.
%
% DO NOT INCLUDE GRAPHICS IN YOUR MANUSCRIPT
% - Figures should be uploaded separately from your manuscript file. 
% - Figures generated using LaTeX should be extracted and removed from the PDF before submission. 
% - Figures containing multiple panels/subfigures must be combined into one image file before submission.
% For figure citations, please use "Fig" instead of "Figure".
% See http://journals.plos.org/plosone/s/figures for PLOS figure guidelines.
%
% Tables should be cell-based and may not contain:
% - spacing/line breaks within cells to alter layout or alignment
% - do not nest tabular environments (no tabular environments within tabular environments)
% - no graphics or colored text (cell background color/shading OK)
% See http://journals.plos.org/plosone/s/tables for table guidelines.
%
% For tables that exceed the width of the text column, use the adjustwidth environment as illustrated in the example table in text below.
%
% % % % % % % % % % % % % % % % % % % % % % % %
%
% -- EQUATIONS, MATH SYMBOLS, SUBSCRIPTS, AND SUPERSCRIPTS
%
% IMPORTANT
% Below are a few tips to help format your equations and other special characters according to our specifications. For more tips to help reduce the possibility of formatting errors during conversion, please see our LaTeX guidelines at http://journals.plos.org/plosone/s/latex
%
% For inline equations, please be sure to include all portions of an equation in the math environment.  For example, x$^2$ is incorrect; this should be formatted as $x^2$ (or $\mathrm{x}^2$ if the romanized font is desired).
%
% Do not include text that is not math in the math environment. For example, CO2 should be written as CO\textsubscript{2} instead of CO$_2$.
%
% Please add line breaks to long display equations when possible in order to fit size of the column. 
%
% For inline equations, please do not include punctuation (commas, etc) within the math environment unless this is part of the equation.
%
% When adding superscript or subscripts outside of brackets/braces, please group using {}.  For example, change "[U(D,E,\gamma)]^2" to "{[U(D,E,\gamma)]}^2". 
%
% Do not use \cal for caligraphic font.  Instead, use \mathcal{}
%
% % % % % % % % % % % % % % % % % % % % % % % % 
%
% Please contact latex@plos.org with any questions.
%
% % % % % % % % % % % % % % % % % % % % % % % %

\documentclass[10pt,letterpaper]{article}
\usepackage[top=0.85in,left=2.75in,footskip=0.75in]{geometry}

% amsmath and amssymb packages, useful for mathematical formulas and symbols
\usepackage{amsmath,amssymb}

% Use adjustwidth environment to exceed column width (see example table in text)
\usepackage{changepage}

% Use Unicode characters when possible
\usepackage[utf8x]{inputenc}

% textcomp package and marvosym package for additional characters
\usepackage{textcomp,marvosym}

% cite package, to clean up citations in the main text. Do not remove.
\usepackage{cite}

% Use nameref to cite supporting information files (see Supporting Information section for more info)
\usepackage{nameref,hyperref}

% line numbers
\usepackage[right]{lineno}

% ligatures disabled
\usepackage{microtype}
\DisableLigatures[f]{encoding = *, family = * }

% color can be used to apply background shading to table cells only
\usepackage[table]{xcolor}

% array package and thick rules for tables
\usepackage{array}

% create "+" rule type for thick vertical lines
\newcolumntype{+}{!{\vrule width 2pt}}

% create \thickcline for thick horizontal lines of variable length
\newlength\savedwidth
\newcommand\thickcline[1]{%
  \noalign{\global\savedwidth\arrayrulewidth\global\arrayrulewidth 2pt}%
  \cline{#1}%
  \noalign{\vskip\arrayrulewidth}%
  \noalign{\global\arrayrulewidth\savedwidth}%
}

% \thickhline command for thick horizontal lines that span the table
\newcommand\thickhline{\noalign{\global\savedwidth\arrayrulewidth\global\arrayrulewidth 2pt}%
\hline
\noalign{\global\arrayrulewidth\savedwidth}}


% Remove comment for double spacing
%\usepackage{setspace} 
%\doublespacing

% Text layout
\raggedright
\setlength{\parindent}{0.5cm}
\textwidth 5.25in 
\textheight 8.75in

% Bold the 'Figure #' in the caption and separate it from the title/caption with a period
% Captions will be left justified
\usepackage[aboveskip=1pt,labelfont=bf,labelsep=period,justification=raggedright,singlelinecheck=off]{caption}
\renewcommand{\figurename}{Fig}

% Use the PLoS provided BiBTeX style
\bibliographystyle{plos2015}

% Remove brackets from numbering in List of References
\makeatletter
\renewcommand{\@biblabel}[1]{\quad#1.}
\makeatother



% Header and Footer with logo
\usepackage{lastpage,fancyhdr,graphicx}
\usepackage{epstopdf}
%\pagestyle{myheadings}
\pagestyle{fancy}
\fancyhf{}
%\setlength{\headheight}{27.023pt}
%\lhead{\includegraphics[width=2.0in]{PLOS-submission.eps}}
\rfoot{\thepage/\pageref{LastPage}}
\renewcommand{\headrulewidth}{0pt}
\renewcommand{\footrule}{\hrule height 2pt \vspace{2mm}}
\fancyheadoffset[L]{2.25in}
\fancyfootoffset[L]{2.25in}
\lfoot{\today}

%% Include all macros below

\newcommand{\lorem}{{\bf LOREM}}
\newcommand{\ipsum}{{\bf IPSUM}}

%% END MACROS SECTION


\begin{document}
\vspace*{0.2in}

% Title must be 250 characters or less.
\begin{flushleft}
{\Large
\textbf\newline{A comparison of models for polygon data, point data or both for malaria mapping} % Please use "sentence case" for title and headings (capitalize only the first word in a title (or heading), the first word in a subtitle (or subheading), and any proper nouns).
}
\newline
% Insert author names, affiliations and corresponding author email (do not include titles, positions, or degrees).
\\
Tim C.D. Lucas*\textsuperscript{1},
et al\textsuperscript{1},
\\
\bigskip
\textbf{1} BDI, Oxford
\\
\bigskip

% Insert additional author notes using the symbols described below. Insert symbol callouts after author names as necessary.
% 
% Remove or comment out the author notes below if they aren't used.
%

% Current address notes



% Use the asterisk to denote corresponding authorship and provide email address in note below.
* timcdlucas@gmail.com

\end{flushleft}
% Please keep the abstract below 300 words
\section*{Abstract}
malaria is bad but decreasing
moving to strategy of elimination country by country
elimination requires maps of low prevalence areas
but maps in low prevalence areas are difficult
sometimes little data
traditional prevalence mapping won't work
we need new sources of data.



% Please keep the Author Summary between 150 and 200 words
% Use first person. PLOS ONE authors please skip this step. 
% Author Summary not valid for PLOS ONE submissions.   
\section*{Author summary}
malaria is bad but decreasing
moving to strategy of elimination country by country
elimination requires maps of low prevalence areas
but maps in low prevalence areas are difficult
sometimes little data
traditional prevalence mapping won't work
we need new sources of data.


\linenumbers

% Use "Eq" instead of "Equation" for equation citations.
%%%%%%%%%%%%%%%%%%%%%%%%%%%%%%%%%%%%%%%%%%%%%%%%%%%%%%%%%%%%%%%%%%%%%%%%%%%%%%%%%%%%%%%%%%%%%%%%%%%%%
\section*{Introduction}
%%%%%%%%%%%%%%%%%%%%%%%%%%%%%%%%%%%%%%%%%%%%%%%%%%%%%%%%%%%%%%%%%%%%%%%%%%%%%%%%%%%%%%%%%%%%%%%%%%%%%

malaria is bad but decreasing
moving to strategy of elimination country by country
elimination requires maps of low prevalence areas
but maps in low prevalence areas are difficult
sometimes little data
traditional prevalence mapping won't work
we need new sources of data.


however surveillance data is sometimes high quality in these areas
pixel level maps from aggregates data is difficult (sturrock, inla, Leon, others)
not much information for covariates 
sturrock used two.
undeveloped area of statistics but see Leon.

Ideal situation is to combine data
benefits of both
but data are on different scales
Ewan says there's a relationship.

here we compared 3 models, point only, polygon only and joint models
we use Madagascar and Indonesia as case studies as they have good surveillance data and good pr data.
we find that...


%%%%%%%%%%%%%%%%%%%%%%%%%%%%%%%%%%%%%%%%%%%%%%%%%%%%%%%%%%%%%%%%%%%%%%%%%%%%%%%%%%%%%%%%%%%%%%%%%%%%%
\section*{Materials and methods}
%%%%%%%%%%%%%%%%%%%%%%%%%%%%%%%%%%%%%%%%%%%%%%%%%%%%%%%%%%%%%%%%%%%%%%%%%%%%%%%%%%%%%%%%%%%%%%%%%%%%%


\subsection*{Malaria data}

We used two data sources that reflect malaria burden; point prevalence surveys and polygon-level, aggregated incidence data.
We selected Madagascar and Indonesia as case examples as they have both good surveillance data and good country wide surveys from approximately the same time.
Given the available data we chose to focus our analysis on 2012-2015 for Indonesia and 2013-2014 for Madagascar.

\subsection*{Prevalence survey data}

The prevalence survey data were extracted from the MAP prevalence survey database [@].
As the prevalence surveys cover different are ranges they were standardisePars using the model from [@].

\subsection*{Polygon incidence data}


\subsection*{Population data}

\subsection*{Covariate data}


\subsection*{The model}

We compared three models 1) a full model with prevalence surveys and aggregated incidence data 2) the submodel with only prevalence data and 3) the submodel with only aggregated incidence data.

The full, joint likelihood model is described as follows. 
Values at the aggregate, polygon level are given the subscript $a$ while pixel or point level below are indexed with $b$.
The polygon case count data, $y_j$ is given a Poisson likelihood

$$y_a \sim \operatorname{Pois}(i_a\mathrm{pop_a})$$

where $i_a$ is the estimated polygon incidence rate and $\mathrm{pop_a}$ is the observed polygon population-at-risk.

The point-level prevalence data, $z_b$, is given a binomial likelihood

$$z_b \sim \operatorname{Binom}(p_b, n_b) $$

where $p_b$ is the estimated prevalence and $n_b$ is the observed survey sample size. 

The two quantities are linked to each other and to the predictor variables by the following system of equations.

$$i_a = \frac{ \sum(i_b \mathrm{pop}_b)}{\sum  \mathrm{pop}_b} $$

$$i_b = \mathrm{p2i}(p_b)$$

where $\mathrm{p2i}$ is a from a model that was published previously. [@cameron2015defining]
After fitting, this model defines a function
$$\mathrm{p2i}: f\left(P\right) = 2.616P - 3.596P^2 + 1.594P^3$$.

The linear predictor of the model, $\eta_b$, is related to prevalence by a typical logit link function.

$$p_b = \operatorname{logit}^{-1}(\eta_b)$$

The linear predictor is composed of an intercept, covariates and a spatial, Gaussian random field.

$$\eta_b = \beta_0 + \beta X  + u(s, \tau, \kappa)$$

The spatial effect $u(s, \tau, \kappa)$ has a Mat\'ern covariance function and two hyper parameters. blah.

Finally, we complete the model by setting priors on the parameters $\beta_0, \beta, \tau$ and $\kappa$.

We set

$$\log(\tau)\sim \operatorname{ Norm}(2, 6)$$

which corresponds to a mean nominal range $r = \frac{\tau}{\sqrt{2}}$ of one fifth of the range of the study area.

We set
$$\log(\kappa)\sim \operatorname{ Norm}(2, 6)$$

for some reason.
Finally, we set regularising priors on the regression coefficients

$$\beta_i \sim \operatorname{ Norm}(0, \sigma)$$

where we try the values for $\sigma$, 0.2, 1 and 2.

Given this setup, we get predictions of prevalence and incidence simultaneously whether we used both data types or just one.


\subsection*{Experiments}

To compare the three models we used two cross validation schemes. 
In the first, the combined data set of prevalence and incidence data was split into five cross-validation folds stratified by data type.
This reflects real world data where we sometimes have incidence data but no prevalence data and visa versa.
In the second validation scheme the incidence data was split into five spatial cross-validation folds (see Figure~\ref{fig:data}b).
Then, any prevalence data within the hold out incidence polygons was also withheld.
This scheme is testing the models' ability to predict into new areas with no information from the spatial random field.

In both cases we examined performance metrics for both the withheld prevalence data and the withheld incidence data.
As there is no good way of combining predictive error from both types of data into one performance metric, we considered performance separately throughout.
Our main performance metric was Pearson's correlation.
However, we also considered Spearman's correlation.
If the prevalence-incidence relationship is poorly estimated, this will strongly affect the Pearson's correlation while the Spearman's correlation will be relatively robust to this source of poor model performance.





% Results and Discussion can be combined.
%%%%%%%%%%%%%%%%%%%%%%%%%%%%%%%%%%%%%%%%%%%%%%%%%%%%%%%%%%%%%%%%%%%%%%%%%%%%%%%%%%%%%%%%%%%%%%%%%%%%%
\section*{Results}
%%%%%%%%%%%%%%%%%%%%%%%%%%%%%%%%%%%%%%%%%%%%%%%%%%%%%%%%%%%%%%%%%%%%%%%%%%%%%%%%%%%%%%%%%%%%%%%%%%%%%


\begin{figure}[!h]
% to be removed before submission
\includegraphics{figures/mdg_points_and_polygons.png}
\caption{{\bf Bold the figure title.}
Figure caption text here, please use this space for the figure panel descriptions instead of using subfigure commands. A: Lorem ipsum dolor sit amet. B: Consectetur adipiscing elit.}
\label{fig1}
\end{figure}



%%%%%%%%%%%%%%%%%%%%%%%%%%%%%%%%%%%%%%%%%%%%%%%%%%%%%%%%%%%%%%%%%%%%%%%%%%%%%%%%%%%%%%%%%%%%%%%%%%%%%
\section*{Discussion}
%%%%%%%%%%%%%%%%%%%%%%%%%%%%%%%%%%%%%%%%%%%%%%%%%%%%%%%%%%%%%%%%%%%%%%%%%%%%%%%%%%%%%%%%%%%%%%%%%%%%%




%%%%%%%%%%%%%%%%%%%%%%%%%%%%%%%%%%%%%%%%%%%%%%%%%%%%%%%%%%%%%%%%%%%%%%%%%%%%%%%%%%%%%%%%%%%%%%%%%%%%%
\section*{Conclusion}
%%%%%%%%%%%%%%%%%%%%%%%%%%%%%%%%%%%%%%%%%%%%%%%%%%%%%%%%%%%%%%%%%%%%%%%%%%%%%%%%%%%%%%%%%%%%%%%%%%%%%





\section*{Supporting information}

% Include only the SI item label in the paragraph heading. Use the \nameref{label} command to cite SI items in the text.
\paragraph*{S1 Fig.}
\label{S1_Fig}
{\bf Bold the title sentence.} Add descriptive text after the title of the item (optional).


\section*{Acknowledgments}
Thanks everyone.
\nolinenumbers

% Either type in your references using
% \begin{thebibliography}{}
% \bibitem{}
% Text
% \end{thebibliography}
%
% or
%
% Compile your BiBTeX database using our plos2015.bst
% style file and paste the contents of your .bbl file
% here. See http://journals.plos.org/plosone/s/latex for 
% step-by-step instructions.
% 
\begin{thebibliography}{10}


\end{thebibliography}



\end{document}

