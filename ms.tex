% Template for PLoS
% Version 3.5 March 2018
%
% % % % % % % % % % % % % % % % % % % % % %
%
% -- IMPORTANT NOTE
%
% This template contains comments intended 
% to minimize problems and delays during our production 
% process. Please follow the template instructions
% whenever possible.
%
% % % % % % % % % % % % % % % % % % % % % % % 
%
% Once your paper is accepted for publication, 
% PLEASE REMOVE ALL TRACKED CHANGES in this file 
% and leave only the final text of your manuscript. 
% PLOS recommends the use of latexdiff to track changes during review, as this will help to maintain a clean tex file.
% Visit https://www.ctan.org/pkg/latexdiff?lang=en for info or contact us at latex@plos.org.
%
%
% There are no restrictions on package use within the LaTeX files except that 
% no packages listed in the template may be deleted.
%
% Please do not include colors or graphics in the text.
%
% The manuscript LaTeX source should be contained within a single file (do not use \input, \externaldocument, or similar commands).
%
% % % % % % % % % % % % % % % % % % % % % % %
%
% -- FIGURES AND TABLES
%
% Please include tables/figure captions directly after the paragraph where they are first cited in the text.
%
% DO NOT INCLUDE GRAPHICS IN YOUR MANUSCRIPT
% - Figures should be uploaded separately from your manuscript file. 
% - Figures generated using LaTeX should be extracted and removed from the PDF before submission. 
% - Figures containing multiple panels/subfigures must be combined into one image file before submission.
% For figure citations, please use "Fig" instead of "Figure".
% See http://journals.plos.org/plosone/s/figures for PLOS figure guidelines.
%
% Tables should be cell-based and may not contain:
% - spacing/line breaks within cells to alter layout or alignment
% - do not nest tabular environments (no tabular environments within tabular environments)
% - no graphics or colored text (cell background color/shading OK)
% See http://journals.plos.org/plosone/s/tables for table guidelines.
%
% For tables that exceed the width of the text column, use the adjustwidth environment as illustrated in the example table in text below.
%
% % % % % % % % % % % % % % % % % % % % % % % %
%
% -- EQUATIONS, MATH SYMBOLS, SUBSCRIPTS, AND SUPERSCRIPTS
%
% IMPORTANT
% Below are a few tips to help format your equations and other special characters according to our specifications. For more tips to help reduce the possibility of formatting errors during conversion, please see our LaTeX guidelines at http://journals.plos.org/plosone/s/latex
%
% For inline equations, please be sure to include all portions of an equation in the math environment.  For example, x$^2$ is incorrect; this should be formatted as $x^2$ (or $\mathrm{x}^2$ if the romanized font is desired).
%
% Do not include text that is not math in the math environment. For example, CO2 should be written as CO\textsubscript{2} instead of CO$_2$.
%
% Please add line breaks to long display equations when possible in order to fit size of the column. 
%
% For inline equations, please do not include punctuation (commas, etc) within the math environment unless this is part of the equation.
%
% When adding superscript or subscripts outside of brackets/braces, please group using {}.  For example, change "[U(D,E,\gamma)]^2" to "{[U(D,E,\gamma)]}^2". 
%
% Do not use \cal for caligraphic font.  Instead, use \mathcal{}
%
% % % % % % % % % % % % % % % % % % % % % % % % 
%
% Please contact latex@plos.org with any questions.
%
% % % % % % % % % % % % % % % % % % % % % % % %

\documentclass[10pt,letterpaper]{article}
\usepackage[top=0.85in,left=2.75in,footskip=0.75in]{geometry}

% amsmath and amssymb packages, useful for mathematical formulas and symbols
\usepackage{amsmath,amssymb}

% Use adjustwidth environment to exceed column width (see example table in text)
\usepackage{changepage}

% Use Unicode characters when possible
\usepackage[utf8x]{inputenc}

% textcomp package and marvosym package for additional characters
\usepackage{textcomp,marvosym}

% cite package, to clean up citations in the main text. Do not remove.
\usepackage{cite}

% Use nameref to cite supporting information files (see Supporting Information section for more info)
\usepackage{nameref,hyperref}

% line numbers
\usepackage[right]{lineno}

% ligatures disabled
\usepackage{microtype}
\DisableLigatures[f]{encoding = *, family = * }

% color can be used to apply background shading to table cells only
\usepackage[table]{xcolor}

% array package and thick rules for tables
\usepackage{array}

% multirow for multiple row tables. 
\usepackage{multirow}

% create "+" rule type for thick vertical lines
\newcolumntype{+}{!{\vrule width 2pt}}

% create \thickcline for thick horizontal lines of variable length
\newlength\savedwidth
\newcommand\thickcline[1]{%
  \noalign{\global\savedwidth\arrayrulewidth\global\arrayrulewidth 2pt}%
  \cline{#1}%
  \noalign{\vskip\arrayrulewidth}%
  \noalign{\global\arrayrulewidth\savedwidth}%
}

% \thickhline command for thick horizontal lines that span the table
\newcommand\thickhline{\noalign{\global\savedwidth\arrayrulewidth\global\arrayrulewidth 2pt}%
\hline
\noalign{\global\arrayrulewidth\savedwidth}}


% Remove comment for double spacing
%\usepackage{setspace} 
%\doublespacing

% Text layout
\raggedright
\setlength{\parindent}{0.5cm}
\textwidth 5.25in 
\textheight 8.75in

% Bold the 'Figure #' in the caption and separate it from the title/caption with a period
% Captions will be left justified
\usepackage[aboveskip=1pt,labelfont=bf,labelsep=period,justification=raggedright,singlelinecheck=off]{caption}
\renewcommand{\figurename}{Fig}

% Use the PLoS provided BiBTeX style
\bibliographystyle{plos2015}

% Remove brackets from numbering in List of References
\makeatletter
\renewcommand{\@biblabel}[1]{\quad#1.}
\makeatother



% Header and Footer with logo
\usepackage{lastpage,fancyhdr,graphicx}
\usepackage{epstopdf}
%\pagestyle{myheadings}
\pagestyle{fancy}
\fancyhf{}
%\setlength{\headheight}{27.023pt}
%\lhead{\includegraphics[width=2.0in]{PLOS-submission.eps}}
\rfoot{\thepage/\pageref{LastPage}}
\renewcommand{\headrulewidth}{0pt}
\renewcommand{\footrule}{\hrule height 2pt \vspace{2mm}}
\fancyheadoffset[L]{2.25in}
\fancyfootoffset[L]{2.25in}
\lfoot{\today}

%% Include all macros below

\newcommand{\lorem}{{\bf LOREM}}
\newcommand{\ipsum}{{\bf IPSUM}}

%% END MACROS SECTION


\begin{document}
\vspace*{0.2in}

% Title must be 250 characters or less.
\begin{flushleft}
{\Large
\textbf\newline{Joint modelling of aggregated incidence data and point prevalence surveys for malaria mapping} % Please use "sentence case" for title and headings (capitalize only the first word in a title (or heading), the first word in a subtitle (or subheading), and any proper nouns).
}
\newline
% Insert author names, affiliations and corresponding author email (do not include titles, positions, or degrees).
\\
Tim C.D. Lucas*\textsuperscript{1}, Anita Nandi\textsuperscript{1}, Michele Nguyen\textsuperscript{1}, Susan Rumisha \textsuperscript{1}, Ewan Cameron\textsuperscript{1}, Pete Gething\textsuperscript{1} and Dan Weiss\textsuperscript{1},
\\
\bigskip
\textbf{1} BDI, Oxford
\\
\bigskip

% Insert additional author notes using the symbols described below. Insert symbol callouts after author names as necessary.
% 
% Remove or comment out the author notes below if they aren't used.
%

% Current address notes



% Use the asterisk to denote corresponding authorship and provide email address in note below.
* timcdlucas@gmail.com

\end{flushleft}
% Please keep the abstract below 300 words
\section*{Abstract}
malaria is bad but decreasing
moving to strategy of elimination country by country
elimination requires maps of low prevalence areas
but maps in low prevalence areas are difficult
sometimes little data
traditional prevalence mapping won't work
we need new sources of data.

Despite major reductions in incidence over the last fifteen years, malaria still causes millions of deaths and loads of DALYs annually \cite{gbd2016}.
Elimination of malaria, country-by-county has become a central goal in the global malaria strategy \cite{wmr2016}.
Elimination of malaria requires high resolution maps of malaria risk, in low prevalence areas in order to optimise interventions and minimise costs \cite{pete_elim_paper}.

% Please keep the Author Summary between 150 and 200 words
% Use first person. PLOS ONE authors please skip this step. 
% Author Summary not valid for PLOS ONE submissions.   
\section*{Author summary}
malaria is bad but decreasing
moving to strategy of elimination country by country
elimination requires maps of low prevalence areas
but maps in low prevalence areas are difficult
sometimes little data
traditional prevalence mapping won't work
we need new sources of data.


\linenumbers

% Use "Eq" instead of "Equation" for equation citations.
%%%%%%%%%%%%%%%%%%%%%%%%%%%%%%%%%%%%%%%%%%%%%%%%%%%%%%%%%%%%%%%%%%%%%%%%%%%%%%%%%%%%%%%%%%%%%%%%%%%%%
\section*{Introduction}
%%%%%%%%%%%%%%%%%%%%%%%%%%%%%%%%%%%%%%%%%%%%%%%%%%%%%%%%%%%%%%%%%%%%%%%%%%%%%%%%%%%%%%%%%%%%%%%%%%%%%

Global malaria incidence has decreased dramatically over the last 20 years \cite{gbd2017}.
This has been accompanied by a strategic shift towards aiming for elimination in low burden countries \cite{wmr?}.
Accurate, high-resolution maps of malaria risk are vital in these countries in the elimination and pre-elimination phases \cite{Gething map op-ed?}.
However, mapping malaria in low burden countries presents new challenges as traditional mapping of prevalence using cluster-level surveys and model-based geostatistics are not necessarily effective in these areas \cite {Gething op-ed, sturrock}.
In low burden areas, very large sample sizes are needed before a prevalence survey is informative.
Furthermore, large, countrywide surveys of prevalence, such as DHS or MIS, are rare in low burden countries due to costs and percieved low utility \cite{dunno}.
We therefore need new data sources and associated statistical methods.

Routine surveillance data, of disease case counts, is becoming more widely available and of better quality.
This data is commonly aggregated, reporting disease case counts that occurred within a certain geographic region defined by a polygon.
This type of data is more sensitive than prevalence surveys in low transmission areas as three entire health system is being used to passively monitor disease risk.
Routine surveillance is also often timely, with reports being released annually or monthly, while large prevalence surveys are performed on a more decadely time scale.
However, the unstructured nature of the data collection means that dealing with biases and data gaps is an ongoing research priority \cite{battlets, cibulski}.

%however surveillance data is sometimes high quality in these areas
%pixel level maps from aggregates data is difficult (sturrock, inla, Leon, others)
%not much information for covariates 
sturrock used two.
%undeveloped area of statistics but see Leon.

Furthermore, using aggregated data to estimate malaria risk for high resolution maps presents further modelling difficulties \cite{sturrock, inla, Leon, others}.
The aggregated nature of the data means that each data point does not necessarily contain much information, especially if the case counts are aggregated over a large or heterogenous area.
For example, \cite{sturrock} use only two covariates do to the lack of information in their data.

A model that could combine point surveys and aggregated surveillance has great potential if it could leverage the strengths of both aggregated surveillance data and point survey data.
One major barrier to this modelling approved us that these two data sources are on different scales: point surveys are a measurement of prevalence $\lbrack 0, 1\rbrack$ while routine surveillance measures incidence $\lbrack 0, \infty\rbrack$.
One approach to combining these datasets is to use a seperately estimated model that maps one scale to another such as \cite{cameron2015}.

%Ideal situation is to combine data
benefits of both
%but data are on different scales
%Ewan says there's a relationship.

Here we formulate a joint model that combines aggregated surveillance data and point prevalence surveys of \emph{Plasmodium falciparum}.
We compare the joint model to models fitted with only one type of data.
As case studies we use three countries with good routine surveillance data as well as good coverage of prevalence point surveys.


%here we compared 3 models, point only, polygon only and joint models
%we use Madagascar and Indonesia as case studies as they have good %surveillance data and good pr data.
we find that...


%%%%%%%%%%%%%%%%%%%%%%%%%%%%%%%%%%%%%%%%%%%%%%%%%%%%%%%%%%%%%%%%%%%%%%%%%%%%%%%%%%%%%%%%%%%%%%%%%%%%%
\section*{Materials and methods}
%%%%%%%%%%%%%%%%%%%%%%%%%%%%%%%%%%%%%%%%%%%%%%%%%%%%%%%%%%%%%%%%%%%%%%%%%%%%%%%%%%%%%%%%%%%%%%%%%%%%%


\subsection*{Malaria data}

We used two data sources that reflect malaria burden; point prevalence surveys and polygon-level, aggregated incidence data.
We selected Madagascar and Indonesia as case examples as they have both good surveillance data and good country wide surveys from approximately the same time.
Given the available data we chose to focus our analysis on 2012-2015 for Indonesia and 2013-2014 for Madagascar.

\subsection*{Prevalence survey data}

The prevalence survey data were extracted from the MAP prevalence survey database \cite{}.
As the prevalence surveys cover different are ranges they were standardised using the model from \cite{}.

\subsection*{Polygon incidence data}


\subsection*{Population data}

\subsection*{Covariate data}

We considered an initial suite of environmental and anthropological covariates, at a resolution of approximately $5 \times 5$ kilometres that included land surface temperature annual mean and standard deviation, enhanced vegetation index, mosquito temperature suitability index, elevation, tassel cap brightness, tassel cap wetness, accessibility to cities, night lights and proportion of urban land cover.
Elevation, land surface temperature standard deviation, accessibility to cities and night lights were all log transformed to reduce skewness.
After preliminary analyses, tassel cap brightness and urban land cover were removed as they were highly correlated with other variables.
The covariates are standardised to have a mean of zero and a standard deviation of one.

\subsection*{The model}

We compared three models 1) a full model with prevalence surveys and aggregated incidence data 2) the submodel with only prevalence data and 3) the submodel with only aggregated incidence data.

The full, joint likelihood model is described as follows. 
Values at the aggregate, polygon level are given the subscript $a$ while pixel or point level below are indexed with $b$.
The polygon case count data, $y_j$ is given a Poisson likelihood

$$y_a \sim \operatorname{Pois}(i_a\mathrm{pop_a})$$

where $i_a$ is the estimated polygon incidence rate and $\mathrm{pop_a}$ is the observed polygon population-at-risk.

The point-level prevalence data, $z_b$, is given a binomial likelihood

$$z_b \sim \operatorname{Binom}(p_b, n_b) $$

where $p_b$ is the estimated prevalence and $n_b$ is the observed survey sample size. 

The two quantities are linked to each other and to the predictor variables by the following system of equations.

$$i_a = \frac{ \sum(i_b \mathrm{pop}_b)}{\sum  \mathrm{pop}_b} $$

$$i_b = \mathrm{p2i}(p_b)$$

where $\mathrm{p2i}$ is from a model that was published previously. \cite{cameron2015defining}
After fitting, this model defines a function
$$\mathrm{p2i}: f\left(P\right) = 2.616P - 3.596P^2 + 1.594P^3$$.

The linear predictor of the model, $\eta_b$, is related to prevalence by a typical logit link function.

$$p_b = \operatorname{logit}^{-1}(\eta_b)$$


The form of the link function means we get predictions of prevalence and incidence simultaneously whether we used both data types or just one.
The linear predictor is composed of an intercept, covariates a spatial, Gaussian random field two iid random effects.

$$\eta_b = \beta_0 + \beta X  + u(s, \rho, \sigma_u) + v_j(\sigma_v) + w_i(\sigma_w)$$

The Gaussian spatial effect $u(s, \rho, \sigma_u)$ has a Mat\'ern covariance function and two hyper parameters: $\rho$, the nominal range (beyond which correlation is $< 0.1$) and $\sigma_u$, the marginal standard deviation.
The first iid random effect, $v_j \sim \operatorname{Norm}(0, \sigma_v)$,  was grouped by polygon, with all pixels and point surveys within polygon $j$ being grouped together.
This random effect modelled both missing covariates and extra-Poisson sampling error. 
The second iid random effect, $w_i \sim \operatorname{Norm}(0, \sigma_w)$, was applied to each point survey.
This effect modelled extra-binomial sampling noise.
As such, this random effect is not included in the predicted uncertainty in the incidence or prevalence layers.
However, it is included in predictions of out-of-sample point surveys.

Finally, we complete the model by setting priors on the parameters $\beta_0, \beta, \rho$ and $\sigma_u$, $\sigma_v$ and $\sigma_w$.
We assigned $\rho$ and $\sigma_u$ a joint penalised complexity prior \ref{notSimpson} such that $P(\rho < 3) = 0.00001$ and $P(\sigma_u > 1) = 0.00001$.
We believe that a large proportion of the variance of malaria prevalence and incidence cannot be explained by a linear combination of the covariates selected, so we set this prior such that the randome field could explain most of the range of the data.

We assigned $\sigma_v$ a penalised complexity prior such that $P(\sigma_v > 0.05) = 0.0000001$
This was based on a comparison of the variance of Poisson random variables, with rates given by the number of cases observed, and an independently derived upper and lower bound for three case counts \ref{cibulskis}.
We found that an iid effect with a standard deviation of 0.05 would be able to account for the discrepancy between the assumed Poisson error and the independently derived error.
We assigned $\sigma_w$ a penalised complexity prior such that $P(\sigma_w > 0.3) = 0.0000001$. 
This was chosen by finding the maximum difference in prevalence between survey points (with a sample size greater than 500) within the same raster pixel.
The differences between points within the same pixel can only be accounted for by the binomial error and this iid effect.
Given that the error on a prevalence estimate with sample size greater than 500 is quite small, the iid effect needs to be able to explain this difference.

Finally, we set regularising priors on the regression coefficients $\beta_i \sim \operatorname{ Norm}(0, 0.04)$. 
Given the standardised covariates, a regression coefficient from the 95\% IQR of this distribution, and an intercept of 3, I've covariate would be able to predict prevalences between 0.004 and 0.27. 
This prior encodes our belief that the full range of malaria transmission can not be explained by a single covariate and our desire to regularised our model.


\subsection*{Experiments}

To compare the three models we used two cross validation schemes. 
In the first, the combined data set of prevalence and incidence data was split into five cross-validation folds stratified by data type.
This reflects real world data where we sometimes have incidence data but no prevalence data and visa versa.
Thus this cross-validation scheme can be seen as asking whether aggregated incidence data provides a viable or preferable alternative to using just prevalence data as in previous modelling efforts \cite{bhatt2016nature}.

%figure 1.cross validation. %%columns country. rows: CV  Indonesia and Senegal only.

\begin{figure}[!h]
% to be removed before submission
\centering
\begin{minipage}{0.45\textwidth} \centering 
\includegraphics[width = 0.9\textwidth]{figures/idn_cv_random.png} %\caption{Indonesia random crossvalidation} 
\end{minipage}
\begin{minipage}{0.45\textwidth} \centering 
\includegraphics[width = 0.9\textwidth]{figures/idn_cv_spatial.png} %\caption{Indonesia spatial crossvalidation} 
\end{minipage}

\caption{{\bf Input incidence and prevalence data and predicted incidence maps. } The incidence and prevalence data (left) and (spatially) out-of-sample predictions (right) for Indonesia (top) and Senagal (bottom).}
\label{fig1}
\end{figure}





In the second validation scheme the data was split into five spatial cross-validation folds (see Figure~\ref{fig:data}b).
The data was divided by combining the point locations and the centroids of the polygons and performing k means clustering.
This scheme is testing the models' ability to predict into new areas with little information from the spatial random field.

In both cases we examined performance metrics for both the withheld prevalence data and the withheld incidence data.
As there is no good way of combining predictive error from both types of data into one performance metric, we considered performance separately throughout.
Our main performance metric was Pearson's correlation.
However, we also considered Spearman's correlation.
If the prevalence-incidence relationship is poorly estimated, this will strongly affect the Pearson's correlation while the Spearman's correlation will be relatively robust to this source of poor model performance.





% Results and Discussion can be combined.
%%%%%%%%%%%%%%%%%%%%%%%%%%%%%%%%%%%%%%%%%%%%%%%%%%%%%%%%%%%%%%%%%%%%%%%%%%%%%%%%%%%%%%%%%%%%%%%%%%%%%
\section*{Results and Discussion}
%%%%%%%%%%%%%%%%%%%%%%%%%%%%%%%%%%%%%%%%%%%%%%%%%%%%%%%%%%%%%%%%%%%%%%%%%%%%%%%%%%%%%%%%%%%%%%%%%%%%%




%figure 2. data and predicted incidence maps. Indonesia and Senegal only. spatial CV only.

%figure 3 predicted Vs observed. colour by %country facet by prev inc and cv scheme? best model only.

\begin{figure}[!h]
% to be removed before submission
\includegraphics[width = 0.9\textwidth]{figures/pred_obs_map.png}
\caption{{\bf Input incidence and prevalence data and predicted incidence maps. } The incidence and prevalence data (left) and (spatially) out-of-sample predictions (right) for Indonesia (top) and Senagal (bottom).}
\label{predobsmap}
\end{figure}






\begin{table}[!ht]
\begin{adjustwidth}{-2.25in}{0in} % Comment out/remove adjustwidth environment if table fits in text column.
\centering
\caption{
{\bf Summary of out-of-sample accuracy for random crossvalidation experiment.}}
\begin{tabular}{llllll}
\hline
{\bf Holdout data} & {\bf Metric} & {\bf Country} &  {\bf Points} & {\bf Polygons} & {\bf Joint} \\
\thickhline 
Incidence & Pearson & Indonesia & 0.1 & 0.2 & {\bf 0.3}\\
                   && Madagascar & 0.1 & 0.2 & {\bf 0.3}\\
                   && Senagal & 0.1 & 0.2 & {\bf 0.3}\vspace{1mm}\\ 
          & Spearman & Indonesia & 0.1 & 0.2 & {\bf 0.3}\\
                   && Madagascar & 0.1 & 0.2 & {\bf 0.3}\\
                   && Senagal & 0.1 & 0.2 & {\bf 0.3}\vspace{3mm} \\ 
Prevalence & Pearson & Indonesia & 0.1 & 0.2 & {\bf 0.3}\\
                   && Madagascar & 0.1 & 0.2 & {\bf 0.3}\\
                   && Senagal & 0.1 & 0.2 & {\bf 0.3}\vspace{1mm}\\ 
            & Spearman & Indonesia & 0.1 & 0.2 & {\bf 0.3}\\
                   && Madagascar & 0.1 & 0.2 & {\bf 0.3}\\
                   && Senagal & 0.1 & 0.2 & {\bf 0.3}\\
\end{tabular}
\begin{flushleft}
Correlation (Spearman's and Pearson's) of predicted incidence rate and prevalence against out-of-sample observed data for three countries.
The data were split randomly into 10 groups stratified by data type (incidence polygon and prevalence points).
\end{flushleft}
\label{table1}
\end{adjustwidth}
\end{table}


\begin{table}[!ht]
\begin{adjustwidth}{-2.25in}{0in} % Comment out/remove adjustwidth environment if table fits in text column.
\centering
\caption{
{\bf Summary of out-of-sample accuracy for spatial crossvalidation experiment.}}
\begin{tabular}{llllll}
\hline
{\bf Holdout data} & {\bf Metric} & {\bf Country} &  {\bf Points} & {\bf Polygons} & {\bf Joint} \\
\thickhline 
Incidence & Pearson & Indonesia & 0.1 & 0.2 & {\bf 0.3}\\
                   && Madagascar & 0.1 & 0.2 & {\bf 0.3}\\
                   && Senagal & 0.1 & 0.2 & {\bf 0.3}\vspace{1mm}\\ 
          & Spearman & Indonesia & 0.1 & 0.2 & {\bf 0.3}\\
                   && Madagascar & 0.1 & 0.2 & {\bf 0.3}\\
                   && Senagal & 0.1 & 0.2 & {\bf 0.3}\vspace{3mm} \\ 
Prevalence & Pearson & Indonesia & 0.1 & 0.2 & {\bf 0.3}\\
                   && Madagascar & 0.1 & 0.2 & {\bf 0.3}\\
                   && Senagal & 0.1 & 0.2 & {\bf 0.3}\vspace{1mm}\\ 
            & Spearman & Indonesia & 0.1 & 0.2 & {\bf 0.3}\\
                   && Madagascar & 0.1 & 0.2 & {\bf 0.3}\\
                   && Senagal & 0.1 & 0.2 & {\bf 0.3}\\
\end{tabular}
\begin{flushleft}
Correlation (Spearman's and Pearson's) of predicted incidence rate and prevalence against out-of-sample observed data for three countries.
The data were split spatiall into 5 groups.
\end{flushleft}
\label{table1}
\end{adjustwidth}
\end{table}





\begin{figure}[!h]
% to be removed before submission
\includegraphics[width = 0.9\textwidth]{figures/joint_idn_cv_log_results}
\caption{{\bf Observed-predicted plots for cross-validated data.}
Error bars include uncertainty of the fitted parameters and from sampling error.
100 realisations of the predictions were calculated taking a draw of all parameters from the posterior and the iid random effect $v_j$ and then sampling from the assumed sampling models (binomial and Poisson) each time.
The 90\% quantiles were them taken and plotted as error bars.
}
\label{predobsscatter}
\end{figure}



%%%%%%%%%%%%%%%%%%%%%%%%%%%%%%%%%%%%%%%%%%%%%%%%%%%%%%%%%%%%%%%%%%%%%%%%%%%%%%%%%%%%%%%%%%%%%%%%%%%%%
\section*{Conclusion}
%%%%%%%%%%%%%%%%%%%%%%%%%%%%%%%%%%%%%%%%%%%%%%%%%%%%%%%%%%%%%%%%%%%%%%%%%%%%%%%%%%%%%%%%%%%%%%%%%%%%%





\section*{Supporting information}

% Include only the SI item label in the paragraph heading. Use the \nameref{label} command to cite SI items in the text.
\paragraph*{S1 Fig.}
\label{S1_Fig}
{\bf Bold the title sentence.} Add descriptive text after the title of the item (optional).


\section*{Acknowledgments}
Thanks everyone.
\nolinenumbers

% Either type in your references using
% \begin{thebibliography}{}
% \bibitem{}
% Text
% \end{thebibliography}
%
% or
%
% Compile your BiBTeX database using our plos2015.bst
% style file and paste the contents of your .bbl file
% here. See http://journals.plos.org/plosone/s/latex for 
% step-by-step instructions.
% 
\bibliography{Malaria} 

%\begin{thebibliography}{10}


%\end{thebibliography}



\end{document}

