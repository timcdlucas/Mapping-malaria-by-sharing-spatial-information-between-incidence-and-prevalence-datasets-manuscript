% Template for PLoS
% Version 3.5 March 2018
%
% % % % % % % % % % % % % % % % % % % % % %
%
% -- IMPORTANT NOTE
%
% This template contains comments intended 
% to minimize problems and delays during our production 
% process. Please follow the template instructions
% whenever possible.
%
% % % % % % % % % % % % % % % % % % % % % % % 
%
% Once your paper is accepted for publication, 
% PLEASE REMOVE ALL TRACKED CHANGES in this file 
% and leave only the final text of your manuscript. 
% PLOS recommends the use of latexdiff to track changes during review, as this will help to maintain a clean tex file.
% Visit https://www.ctan.org/pkg/latexdiff?lang=en for info or contact us at latex@plos.org.
%
%
% There are no restrictions on package use within the LaTeX files except that 
% no packages listed in the template may be deleted.
%
% Please do not include colors or graphics in the text.
%
% The manuscript LaTeX source should be contained within a single file (do not use \input, \externaldocument, or similar commands).
%
% % % % % % % % % % % % % % % % % % % % % % %
%
% -- FIGURES AND TABLES
%
% Please include tables/figure captions directly after the paragraph where they are first cited in the text.
%
% DO NOT INCLUDE GRAPHICS IN YOUR MANUSCRIPT
% - Figures should be uploaded separately from your manuscript file. 
% - Figures generated using LaTeX should be extracted and removed from the PDF before submission. 
% - Figures containing multiple panels/subfigures must be combined into one image file before submission.
% For figure citations, please use "Fig" instead of "Figure".
% See http://journals.plos.org/plosone/s/figures for PLOS figure guidelines.
%
% Tables should be cell-based and may not contain:
% - spacing/line breaks within cells to alter layout or alignment
% - do not nest tabular environments (no tabular environments within tabular environments)
% - no graphics or colored text (cell background color/shading OK)
% See http://journals.plos.org/plosone/s/tables for table guidelines.
%
% For tables that exceed the width of the text column, use the adjustwidth environment as illustrated in the example table in text below.
%
% % % % % % % % % % % % % % % % % % % % % % % %
%
% -- EQUATIONS, MATH SYMBOLS, SUBSCRIPTS, AND SUPERSCRIPTS
%
% IMPORTANT
% Below are a few tips to help format your equations and other special characters according to our specifications. For more tips to help reduce the possibility of formatting errors during conversion, please see our LaTeX guidelines at http://journals.plos.org/plosone/s/latex
%
% For inline equations, please be sure to include all portions of an equation in the math environment.  For example, x$^2$ is incorrect; this should be formatted as $x^2$ (or $\mathrm{x}^2$ if the romanized font is desired).
%
% Do not include text that is not math in the math environment. For example, CO2 should be written as CO\textsubscript{2} instead of CO$_2$.
%
% Please add line breaks to long display equations when possible in order to fit size of the column. 
%
% For inline equations, please do not include punctuation (commas, etc) within the math environment unless this is part of the equation.
%
% When adding superscript or subscripts outside of brackets/braces, please group using {}.  For example, change "[U(D,E,\gamma)]^2" to "{[U(D,E,\gamma)]}^2". 
%
% Do not use \cal for caligraphic font.  Instead, use \mathcal{}
%
% % % % % % % % % % % % % % % % % % % % % % % % 
%
% Please contact latex@plos.org with any questions.
%
% % % % % % % % % % % % % % % % % % % % % % % %

\documentclass[10pt,letterpaper]{article}
\usepackage[top=0.85in,left=2.75in,footskip=0.75in]{geometry}

% amsmath and amssymb packages, useful for mathematical formulas and symbols
\usepackage{amsmath,amssymb}

% Use adjustwidth environment to exceed column width (see example table in text)
\usepackage{changepage}

% Use Unicode characters when possible
\usepackage[utf8x]{inputenc}

% textcomp package and marvosym package for additional characters
\usepackage{textcomp,marvosym}

% cite package, to clean up citations in the main text. Do not remove.
\usepackage{cite}

% Use nameref to cite supporting information files (see Supporting Information section for more info)
\usepackage{nameref,hyperref}

% line numbers
\usepackage[right]{lineno}

% ligatures disabled
\usepackage{microtype}
\DisableLigatures[f]{encoding = *, family = * }

% color can be used to apply background shading to table cells only
\usepackage[table]{xcolor}

% array package and thick rules for tables
\usepackage{array}

% multirow for multiple row tables. 
\usepackage{multirow}



% create "+" rule type for thick vertical lines
\newcolumntype{+}{!{\vrule width 2pt}}

% create \thickcline for thick horizontal lines of variable length
\newlength\savedwidth
\newcommand\thickcline[1]{%
  \noalign{\global\savedwidth\arrayrulewidth\global\arrayrulewidth 2pt}%
  \cline{#1}%
  \noalign{\vskip\arrayrulewidth}%
  \noalign{\global\arrayrulewidth\savedwidth}%
}

% \thickhline command for thick horizontal lines that span the table
\newcommand\thickhline{\noalign{\global\savedwidth\arrayrulewidth\global\arrayrulewidth 2pt}%
\hline
\noalign{\global\arrayrulewidth\savedwidth}}


% Remove comment for double spacing
%\usepackage{setspace} 
%\doublespacing

% Text layout
\raggedright
\setlength{\parindent}{0.5cm}
\textwidth 5.25in 
\textheight 8.75in

% Bold the 'Figure #' in the caption and separate it from the title/caption with a period
% Captions will be left justified
\usepackage[aboveskip=1pt,labelfont=bf,labelsep=period,justification=raggedright,singlelinecheck=off]{caption}
\renewcommand{\figurename}{Fig}

% Use the PLoS provided BiBTeX style
\bibliographystyle{plos2015}

% Remove brackets from numbering in List of References
\makeatletter
\renewcommand{\@biblabel}[1]{\quad#1.}
\makeatother



% Header and Footer with logo
\usepackage{lastpage,fancyhdr,graphicx}
\DeclareGraphicsExtensions{.pdf,.png}   
\usepackage{epstopdf}
%\pagestyle{myheadings}
\pagestyle{fancy}
\fancyhf{}
%\setlength{\headheight}{27.023pt}
%\lhead{\includegraphics[width=2.0in]{PLOS-submission.eps}}
\rfoot{\thepage/\pageref{LastPage}}
\renewcommand{\headrulewidth}{0pt}
\renewcommand{\footrule}{\hrule height 2pt \vspace{2mm}}
\fancyheadoffset[L]{2.25in}
\fancyfootoffset[L]{2.25in}
\lfoot{\today}

%% Include all macros below

\newcommand{\lorem}{{\bf LOREM}}
\newcommand{\ipsum}{{\bf IPSUM}}

\usepackage{xr}
\externaldocument{si}


%% END MACROS SECTION


\begin{document}
\vspace*{0.2in}

% Title must be 250 characters or less.
\begin{flushleft}
{\Large
\textbf\newline{Joint modelling of aggregated incidence data and point prevalence surveys for malaria mapping} % Please use "sentence case" for title and headings (capitalize only the first word in a title (or heading), the first word in a subtitle (or subheading), and any proper nouns).
}
\newline
% Insert author names, affiliations and corresponding author email (do not include titles, positions, or degrees).
\\
Tim C.D. Lucas*\textsuperscript{1}, Anita Nandi\textsuperscript{1}, Michele Nguyen\textsuperscript{1}, Susan Rumisha \textsuperscript{1}, Rosalind Howes \textsuperscript{1}, Katherine E. Battle \textsuperscript{1}, Penelope Hancock \textsuperscript{1}, Andre Python \textsuperscript{1}, Ewan Cameron\textsuperscript{1}, Pete Gething\textsuperscript{1} and Daniel J. Weiss\textsuperscript{1}
\\
\bigskip
\textbf{1} BDI, Oxford
\\
\bigskip

% Insert additional author notes using the symbols described below. Insert symbol callouts after author names as necessary.
% 
% Remove or comment out the author notes below if they aren't used.
%

% Current address notes



% Use the asterisk to denote corresponding authorship and provide email address in note below.
* timcdlucas@gmail.com

\end{flushleft}
% Please keep the abstract below 300 words
\section*{Abstract}
As malaria incidence decreases and more countries move towards elimination, maps of malaria risk in low prevalence areas are becoming increasingly needed.
However, traditional mapping using prevalence point-surveys are often ineffective in low prevalence areas due to low statistical power and a lack of data.
More recently, dissagregation regression models have been developed that can estimate risk at high spatial resolution from surveillance data that reports incidence aggregated to a geographic polygon.
However, there is generally low spatial overlap between prevalence point-surveys and polygon incidence data and therefore models that make use of the spatial information from prevalence point surveys have great potential.
Here, using case studies in Indonesia, Senegal and Madagascar, we compare two methods for incorporating this spatial information into a disaggregation regression model.
The first is a simple model that fits a Gaussian random field to prevalence point-surveys and then uses predictions from the random field as a covariate in the disaggregation regression model.
The second is a more complex, multi-likelihood model that jointly fits a model using prevalence point-surveys and  polygon incidence data.
We find that the simple model generally performs better than a baseline disaggregation model while the joint model sometimes performs better than baseline but often worse.
Our results demonstrate that a simple method can be used to combine spatial information from prevalence point-surveys and polygon incdience data and that more generally, combining these types of data improves estimates of malaria incidence.



% Please keep the Author Summary between 150 and 200 words
% Use first person. PLOS ONE authors please skip this step. 
% Author Summary not valid for PLOS ONE submissions.   
\section*{Author summary}
Todo


\linenumbers

% Use "Eq" instead of "Equation" for equation citations.
%%%%%%%%%%%%%%%%%%%%%%%%%%%%%%%%%%%%%%%%%%%%%%%%%%%%%%%%%%%%%%%%%%%%%%%%%%%%%%%%%%%%%%%%%%%%%%%%%%%%%
\section*{Introduction}
%%%%%%%%%%%%%%%%%%%%%%%%%%%%%%%%%%%%%%%%%%%%%%%%%%%%%%%%%%%%%%%%%%%%%%%%%%%%%%%%%%%%%%%%%%%%%%%%%%%%%


% Glossary
% Models:
%   Joint model
%   Polygon-only model
%   Point-only model
% Data:
%   Prevalence point-surveys
%   Polygon incidence
% Variables:
%   Pixel-level prevalence
%   Polygon-level incidence
%   Pixel-level incidence

Global malaria incidence has decreased dramatically over the last 20 years \cite{abajobir2017global, bhatt2015effect}.
This decrease has been accompanied by a strategic shift towards aiming for elimination in low incidence countries \cite{world2016world, newby2016path}.
Accurate, high-resolution maps of malaria risk are vital in countries in the elimination and pre-elimination phases \cite{sturrock2016mapping, cohen2017mapping}.
However, mapping malaria in low burden countries presents new challenges as traditional mapping of prevalence \cite{gething2011new, bhatt2017improved, gething2012long, bhatt2015effect} using cluster-level surveys and model-based geostatistics are not necessarily effective in these areas \cite{sturrock2016mapping, sturrock2014fine}.
In low burden areas, very large sample sizes are needed before a prevalence survey is informative because so few individuals have parasitaemia that most sample points will have no cases.
Routine surveillance data, of malaria case counts, is becoming more widely available and of better quality \cite{sturrock2016mapping, ohrt2015information, cibulskis2011worldwide}.
This polygon incidence data can be more sensitive than prevalence point-surveys in low transmission areas as the entire public health system is being used to passively monitor disease risk \cite{cibulskis2011worldwide}.

%Despite major reductions in incidence over the last fifteen years, malaria still causes millions of deaths and loads of DALYs annually \cite{abajobir2017global}.
%Elimination of malaria, country-by-county has become a central goal in the global malaria strategy \cite{world2016world}.
%Elimination of malaria requires high resolution maps of malaria risk, in low prevalence areas in order to optimise interventions and minimise costs \cite{sturrock2016mapping}.

%cohen2017mapping Different metrics are useful for different things. Most maps use prevalence. Govts do use maps.

%however surveillance data is sometimes high quality in these areas
%pixel level maps from aggregates data is difficult (sturrock2014fine, wilson2017pointless, law2018variational, others)
%not much information for covariates 
%sturrock used two.
%undeveloped area of statistics but see Leon.

Disaggregation regression methods have been proposed as a way to model malaria burden using polygon-level incidence \cite{sturrock2014fine, wilson2017pointless, law2018variational, taylor2017continuous, li2012log}.
However using polygon incidence data to estimate malaria risk for high resolution maps presents several noteworthy modelling difficulties. 
Firstly, especially when modelling at continental scales \cite{weiss2019mapping, battle2019mapping}, there are often some areas with better coverage of prevalence surveys than polygon incidence data.
%Disaggregation models aim to estimate high-resolution maps from aggregated data and are characterised by having a variable number of pixels, and therefore multiple rows of covariate data, per row of response data.
Furthermore, the aggregation of cases over space means that it may be difficult to learn relationships between malaria risk and the environment from these data, especially if the case counts are aggregated over a large or heterogenous area.
Finally, estimates of both prevalence and incidence metrics are useful for policy makers \cite{cohen2017mapping} but typically models are fitted to observations of one metric and then a secondary model is used to convert to the other \emph{post hoc} \cite{battle2019mapping, bhatt2015effect}.
Models that combine observations of incidence and prevalence have the potential to provide better estimates than the \emph{post hoc} conversion and could also learn the relationship between different metrics at the same time as making spatial estimates.
One major barrier to a joint modelling approach, however, is that these two data types measure metrics disease transmission on different scales.
Point surveys are a measurement of prevalence in the range $\lbrack 0, 1\rbrack$ that quantify parasite rate at a single moment.
In contrast, routine surveillance measures incidence in the range $\lbrack 0, \infty\rbrack$ as it spans a period of time (e.g., a year) over which individuals can have multiple malaria infections.
%One approach to combining these datasets is to use a seperately estimated model that maps one scale to another such as \cite{cameron2015defining}.

%Ideal situation is to combine data
%benefits of both
%but data are on different scales
%Ewan says there's a relationship.

Here we compare two methods for using spatial information from prevalence surveys to inform a disaggregation model of fitted to polygon incidence data.
The first model fits a Gaussian process to prevalence point surveys then uses predictions from this model as a covariate in the disaggregation model.
Secondly, we formulate a joint model that combines polygon incidence data and prevalence point-surveys of \emph{Plasmodium falciparum} using seperate likelihoods for both data types.
We relate the differing malariometric measures by using a previously estimated relationship within the model \cite{cameron2015defining} which is then adjusted as part of the model fitting process.
We then compare results from the two models with those made using a polygon-only, disaggregation model similar to previous models \cite{sturrock2014fine, wilson2017pointless}.
Both models are fitted to data from Indonesia, Senegal and Madagascar to provide a set of case studies with disparate geographic settings and levels of malaria endemicity.

%here we compared 3 models, point only, polygon only and joint models
%we use Madagascar and Indonesia as case studies as they have good %surveillance data and good pr data.




%%%%%%%%%%%%%%%%%%%%%%%%%%%%%%%%%%%%%%%%%%%%%%%%%%%%%%%%%%%%%%%%%%%%%%%%%%%%%%%%%%%%%%%%%%%%%%%%%%%%%
\section*{Materials and methods}
%%%%%%%%%%%%%%%%%%%%%%%%%%%%%%%%%%%%%%%%%%%%%%%%%%%%%%%%%%%%%%%%%%%%%%%%%%%%%%%%%%%%%%%%%%%%%%%%%%%%%


\subsection*{Malaria data}

We used two data sources that reflect malaria burden; prevalence point-surveys and polygon incidence data.
We selected Indonesia, Senegal and Madagascar as case examples as they all have abundant subnational surveillance data and country wide surveys from approximately the same period.
To minimise temporal effects we selected five years of prevalence point-survey data and one year of polygon incidence data for each country.
For Indonesia we selected polygon incidence data from 2012 that covers 380 regions, and prevalence data from 2010 to 2014 that consists of todo surveys, accounting for todo individuals.
For Senegal we selected 2015 for polygon incidence data (46 regions) and 2013 to 2017 for prevalence data (todo surveys, todo).
Finally, for Madagascar we selected 2013 for polygon incidence (110 regions) and 2011 to 2015 for prevalence (todo surveys, todo individuals) data.


\begin{figure}
% to be removed before submission
\includegraphics[width = 0.6\textwidth]{figures/sen_both_cv12_preds.png}
\caption{{\bf Input incidence data and predicted incidence maps. } 
The incidence (log10) data, predicted log10 incidence from the prevalence GP model for spatially cross-validated out-of-sample polygons (middle) and predicted log10 incidence from the joint model for spatially cross-validated out-of-sample polygons (bottom) for Senegal.
}
\label{predobsmapsen}
\end{figure}



\begin{figure}
% to be removed before submission
\includegraphics[width = 1.1\textwidth]{figures/mdg_both_cv12_preds.png}
\caption{{\bf Input incidence data and predicted incidence maps. } 
The incidence (log10) data, predicted log10 incidence from the prevalence GP model for spatially cross-validated out-of-sample polygons (middle) and predicted log10 incidence from the joint model for spatially cross-validated out-of-sample polygons (bottom) for Senegal.
}
\label{predobsmapmdg}
\end{figure}



\subsection*{Prevalence survey data}

The prevalence point-survey data were extracted from the Malaria Atlas Project prevalence survey database \cite{bhatt2015effect, guerra2007assembling, gething2011new, pfeffer2018ma}.
As the prevalence point-surveys cover different age ranges they were standardised to an age-range of 2--10 using the model from \cite{smith2007standardizing}.
The age standardisation model gives the surveys with zero positive cases a small positive prevalence.

\subsection*{Polygon incidence data}

% (Figures~\ref{predobsmapidn}A and \ref{predobsmapsen}A)

The polygon incidence data were collated from various government reports and adjusted to a standard metric using methods defined in \cite{cibulskis2011worldwide, weiss2019mapping}.
These adjustments account for underreporting of clinical cases due to lack of treatment seeking, missed case reports, and cases that sought medical attention outside the public health systems \cite{battle2016treatment}.
Where species specific reports were given, these were used, and in reports that did not distinguish between species of \emph{Plasmodium} the national estimate of the ratio between \emph{P. falciparum} and \emph{P. vivax} cases were used to estimate \emph{P. falciparum} only cases.
The polygon incidence data for Senegal and Madagascar can be seen in Figures~\ref{predobsmapsen}A and \ref{predobsmapmdg}A)


\subsection*{Population data}

Raster surfaces of population for the years 2005, 2010 and 2015, were created using a hybrid of data from GPWv4 \cite{gpw4} and WorldPop \cite{tatem2017worldpop}, with the latter taking priority for those pixels where both had population data. 
Yearly population rasters were created by linear interpolation of the surrounding five-yearly rasters.
For each year, national population estimates from the UN were raked over the interpolated population surfaces. 


\subsection*{Covariate data}

We considered an initial suite of environmental and anthropological covariates, at a resolution of approximately $5 \times 5$ kilometres at the equator that included land surface temperature annual mean and standard deviation \cite{LST}, enhanced vegetation index \cite{TCB}, mosquito temperature suitability index \cite{weiss2014air}, elevation \cite{SRTMElev}, tassel cap brightness \cite{TCB}, tassel cap wetness \cite{TCB}, accessibility to cities \cite{weiss2018global}, night lights \cite{elvidge2017viirs} and proportion of urban land cover \cite{GUF}. % change to table
Some covariates were log-transformed to remove skewness or removed due to multicollineariy with other predictor variables using the threshold of 0.8. % think we decided this on indonesia data only. Sen is currently 0.81... just a bit messy.
The covariates were standardised to have a mean of zero and a standard deviation of one.



\subsection*{Baseline Disaggregation Model}

Values at the aggregate, polygon level are given the subscript $a$ while pixel or point level variables are indexed with $b$.
The polygon incidence case count data, $y_j$ is given a Poisson likelihood

$$y_a \sim \operatorname{Pois}(i_a\mathrm{pop_a})$$

where $i_a$ is the estimated polygon incidence rate and $\mathrm{pop_a}$ is the observed polygon population-at-risk.


Incidence rate is linked to latent pixel-level incidence, $i_b$, prevalence $p_b$ and to the predictor variables by the following system of equations.

$$i_a = \frac{ \sum_{b \in a}i_b \mathrm{pop}_b}{\sum_{b \in a}\mathrm{pop}_b} $$
Here, $b \in a$ denotes that the summation is over the pixels in polygon $a$. 
Incidence is related to prevalence by
$$i_b = \mathrm{PrevInc}(p_b).$$
Here $\mathrm{PrevInc}$ is a function from a previously fitted model \cite{cameron2015defining} 
$$\mathrm{PrevInc}: f\left(p_b\right) = 2.616p_b - 3.596{p_b}^2 + 1.594{p_b}^3.$$
The linear predictor of the model, $\eta_b$, is related to the latent prevalence scale by a typical logit link function.
$$p_b = \operatorname{logit}^{-1}(\eta_b)$$
The form of this set of link functions means we get predictions of prevalence and incidence simultaneously whether we used both data types or just one.

The linear predictor is composed of an intercept, $b_0$, covariates, $X$, and a vector of regression coefficients $\beta$.
We also include a spatial, Gaussian random field, $u(s, \rho, \sigma_u)$ and a polygon-level iid random effect, $ v_j(\sigma_v)$.
$$\eta_b = \beta_0 + \beta X  + u(s, \rho, \sigma_u) + v_j(\sigma_v) $$
The Gaussian spatial effect $u(s, \rho, \sigma_u)$ has a Mat\'ern covariance function and two hyper parameters: $\rho$, the nominal range (beyond which correlation is $< 0.1$) and $\sigma_u$, the marginal standard deviation.
The iid random effect, $v_j \sim \operatorname{Norm}(0, \sigma_v)$,  was grouped by polygon, with all pixels within polygon $j$ being grouped together.
This random effect modelled both missing covariates and extra-Poisson sampling error.


Finally, we complete the model by setting priors on the parameters $\beta_0, \beta, \rho, \sigma_u$, $\sigma_v$ and $\sigma_w$.
The intercept was given a wide prior, $b_0 \sim \operatorname{Norm}(-2, 4)$ with a mean less than zero as we know \emph{a priori} that these countries have low or medium levels of malaria transmission.
We set independant, regularising priors on the regression coefficients $\beta_i \sim \operatorname{Norm}(0, 0.04)$. 
Given the standardised covariates and an intercept of -3, a regression coefficient from the 95\% interquartile range of this distribution, each covariate would be able to predict prevalences between 0.004 and 0.27. 
This prior encodes our belief that the full range of malaria transmission can not be explained by a single covariate and our desire to regularise the model.
This regularisation is particularly important given the small number of observations in Senegal (n = 46) and Madagascar (n = 110).

We assigned $\rho$ and $\sigma_u$ a joint penalised complexity prior \cite{fuglstad2018constructing} such that $P(\rho < \zeta) = 0.00001$ and $P(\sigma_u > 1) = 0.00001$.
We used different $\zeta$ values for each country: Indonesia $\zeta = 3$, Senegal $\zeta = 1$ and Madagascar $\zeta = 1$.
We believe that a large proportion of the variance of malaria prevalence and incidence cannot be explained by a linear combination of the covariates selected \cite{bhatt2017improved}, so we set this prior such that the random field could explain most of the range of the data.

We assigned $\sigma_v$ a penalised complexity prior \cite{simpson2017penalising} such that $P(\sigma_v > 0.05) = 0.0000001$
This was based on a comparison of the variance of Poisson random variables, with rates given by the number of cases observed, and an independently derived upper and lower bound for the case counts using the approach defined in \cite{cibulskis2011worldwide}.
We found that an iid effect with a standard deviation of 0.05 is able to account for the discrepancy between the assumed Poisson error and the independently derived error.


The models were implemented and fitted in R \cite{R} using Template Model Builder \cite{TMB} which allows a Laplace approximation of the posterior to be calculated.
The hyperparameters are fitted using empirical Bayes whereby the hyperparameters are learned from the data but are treated as point estimates rather than using the full posterior of the hyperparameters.

%\begin{figure}[t!]
%% to be removed before submission
%\centering
%
%\includegraphics[width = 0.9\textwidth]{figures/random_crossvalidation_full.png} %\caption{Indonesia random crossvalidation} 
%
%\caption{{\bf Random cross-validation scheme for Indonesia, Senegal and Madagascar.} The fold for both aggregated incidence data and prevalence point data is shown.}
%\label{fig:cv_random}
%\end{figure}
%
%
%\begin{figure}[!t]
%% to be removed before submission
%\centering
%
%\includegraphics[width = 0.9\textwidth]{figures/spatial_crossvalidation_full.png} %\caption{Indonesia random crossvalidation} 
%
%\caption{{\bf Spatial cross-validation scheme for Indonesia, Senegal and Madagascar.} The fold for both aggregated incidence data and prevalence point data is shown.}
%\label{fig:cv_spatial}
%\end{figure}
%

\subsection*{Prevalence Gaussian process covariate model}

This model (henceforth the prevalence GP model) is the same as the baseline disaggregation model except that it has one extra covariate.
This covariate is created by fitting a Gaussian random field to the prevalence survey data.
For each country we fitted a binomial likelihood, hierarchical Gaussian random field with the same hyperpriors for $\rho$ and $\sigma_u$ as above.
These models were fitted using R-INLA \cite{INLA}.
To be in the correct scale for the dissagregation model, the inverse logit of the predicted Gaussian field (i.e. the linear predictor of the model) was used as the additional covariate.

\subsection*{Full joint model}

The final model is a joint-likelihood model with separate likelihoods for prevalence point surveys and polygon incidence data.
The polygon data are assigned a Poisson likelihood as before but the point-survey data, with positive cases $z_b$, is given a binomial likelihood

$$z_b \sim \operatorname{Binom}(p_b, n_b) $$
where $p_b$ is the estimated prevalence and $n_b$ is the observed survey sample size. 
As this model has both prevalence and incidence data we add a parameter $\alpha$ that modifies the relationship between the two.
$$i_b = \exp(\alpha)\mathrm{PrevInc}(p_b).$$

The only further additions to the baseline model are in the linear predictor which becomes 

$$\eta_b = \beta_0 + \mathbf{1}_p\beta_p +  \beta X  + u(s, \rho, \sigma_u) + v_j(\sigma_v) + w_i(\sigma_w).$$
As well as the global intercept, $\beta_0$, this model has a prevalence survey only intercept $b_p$ where the indicator function, $\mathbf{1}_p$ denotes that this term is zero except when a prevalence point survey is being considered.
The iid random effect, $v_j \sim \operatorname{Norm}(0, \sigma_v)$,  was grouped by polygon, with all point surveys within polygon $j$ being in the same group as polygon $j$.
The second iid random effect, $w_i \sim \operatorname{Norm}(0, \sigma_w)$, was applied to each point survey.
This effect modelled extra-binomial sampling noise.
As such, this random effect is not included in the predicted uncertainty in the incidence or prevalence layers.

We assigned $\sigma_w$ a penalised complexity prior such that $P(\sigma_w > 0.3) = 0.0000001$. 
This was chosen by finding the maximum difference in prevalence between point-surveys (with a sample size greater than 500 individuals) within the same raster pixel.
The differences between points within the same pixel can only be accounted for by the binomial error and this iid effect.
Given that the error on a prevalence estimate with sample size greater than 500 is quite small, the iid effect needs to be able to explain this difference.

Given that the PrevInc relationship is fitted to the best available data, we have fairly strong \emph{a priori} confidence in it.
Therefore, our prior belief is that $\exp(\alpha)$ is close to one (i.e.\thinspace the relationship remains unchanged) and therefore that $\alpha$ is close to zero.
We set our prior that $\alpha \sim \operatorname{Norm}(0, 0.001)$.


\subsection*{Experiments}

To compare the three models we used two cross-validation schemes. 
In the first (random), the incidence data was split into ten cross-validation folds while all the prevalence data was used in each case.
%This cross-validation scheme directly addresses the question of whethe
In the second validation scheme , the incidence data was split into spatial cross-validation folds, using k means clustering on polygon centroids, while again all prevalence points were used in all folds.
The number of folds was seven for Indonesia, five for Senegal and three for Madagascar due to their differing sizes and epidemiological settings.
This scheme is specifically testing whether the joint model can improve predictions by increasing geographic data coverage.
This situation is particularly common in continental scale models where some countries will have large DHS style surveys but little incidence data.


We considered the ability to predict polygon incidence correctly our main objective and our performance metric was mean absolute error (MAE).
As the models are being fitted on data on difference scales we found that observations and predictions were sometimes correlated but shifted from the one-one line (i.e. biased) and therefore correlation metrics were misleading.
% Why not correlation
To assess how well the models were calibrated we considered coverage of the 80\% predictive credible intervals on the hold-out data.




\begin{table}[!t]
\begin{adjustwidth}{-2.25in}{0in} % Comment out/remove adjustwidth environment if table fits in text column.
\centering
\caption{
{\bf Summary of out-of-sample accuracy for all three cross-validation experiments.}}
\begin{tabular}{lllll}
\hline
{\bf Cross-validation} & {\bf Country}  & {\bf Baseline} & {\bf Prev GP} & {\bf Joint} \\
\thickhline 
Random & Indonesia  & 13.95 &  14.09 &  {\bf 13.79}\\
& Senegal  & 12.41 &  {\bf 12.37} &  13.07\\
& Madagascar  & 39.06 &  {\bf 35.82} &  36.36\vspace{3mm}\\
Spatial & Indonesia & {\bf 14.77} &  {\bf 14.77} &  16.46\\
& Senegal  & 13.09 &  {\bf 12.21} &  15.15\\
& Madagascar & 67.73 &  50.38 &  {\bf 44.05}\\
\end{tabular}
\begin{flushleft}
Mean absolute error of predicted incidence rate against out-of-sample observed data for three countries.
\end{flushleft}
\label{table1}
\end{adjustwidth}
\end{table}





% Results and Discussion can be combined.
%%%%%%%%%%%%%%%%%%%%%%%%%%%%%%%%%%%%%%%%%%%%%%%%%%%%%%%%%%%%%%%%%%%%%%%%%%%%%%%%%%%%%%%%%%%%%%%%%%%%%
\section*{Results}
%%%%%%%%%%%%%%%%%%%%%%%%%%%%%%%%%%%%%%%%%%%%%%%%%%%%%%%%%%%%%%%%%%%%%%%%%%%%%%%%%%%%%%%%%%%%%%%%%%%%%

% Random cv

Under the random cross-validation scheme, the prevalence GP model performed best in Senegal and Madagascar while the joint model performed best in Indonesia (Table \ref{table1}).
The differences were relatively small in all three countries.
This lack of strong differences is highlighted by the fact that there are no clear differences in scatter plots of observed and predicted data (Figure \ref{randompredobspolyfacet}).



% spatial 1

Under the spatial a cross-validation scheme, the polygon only model and prevalence GP models performed equal best in Indonesia, the prevalence GP performaced best in Senegal while the joint model performed best in Madagascar (Table \ref{table1}).
In contrast to the random cross-validation results, the differences between models is quite strong.
Furthermore, notable difference can be seen in the scattor plots of observed and predicted values (Figure \ref{spatial2predobspolyfacet}).
In Indonesia it can be seen the the joint model is more strongly biased at low incidence values with many data points being overpredicted.
However, the joint model clearly performs better in Madagascar with the polygon only model particularly struggling to predict high incidence observations correctly.
Out-of-sample predictions, under spatial cross validation, from the prevalence GP model and full joint model can be seen in Figures~\ref{predobsmapsen} and \ref{predobsmapmdg}.

\begin{figure}
% to be removed before submission
\includegraphics[width = 1.05\textwidth]{figures/random_cv_poly_facet}  
\caption{{\bf Observed-predicted plots for random cross-validation experiments.}
A) Indonesia, B) Senegal, C) Madagascar. Square root aggregated incidence (per 1,000).
Results from the standard downscaling model are shown in red, the prevalence GP model is shown in green while the joint model is shown in blue.
The one-one line is shown with a black line and a simple linear regression through the points is shown by a coloured line.
}
\label{randompredobspolyfacet}
\end{figure}


%figure 5, 6. Spat and random cv. PR vs Poly columns, countries as rows, model as colour?


\begin{figure}
% to be removed before submission
\includegraphics[width = 1.05\textwidth]{figures/spatialkeeppr_cv_poly_facet}  
\caption{{\bf Observed-predicted plots for spatial a cross-validation experiments where all prevalence points are used in model fitting.}
A) Indonesia, B) Senegal, C) Madagascar. Square root aggregated incidence (per 1,000).
Results from the standard downscaling model are shown in red, the prevalence GP model is shown in green while the joint model is shown in blue.
The one-one line is shown with a black line and a simple linear regression through the points is shown by a coloured line.
}
\label{spatial2predobspolyfacet}
\end{figure}



\begin{table}[t]
\begin{adjustwidth}{-2.25in}{0in} % Comment out/remove adjustwidth environment if table fits in text column.
\centering
\caption{
{\bf Summary of coverage of 80\% credible intervals.}}
\begin{tabular}{lllll}
\hline
{\bf Cross-validation} & {\bf Country}  & {\bf Baseline} & {\bf Prev GP} & {\bf Joint} \\
\thickhline 
Random & Indonesia  & 0.73 &  0.72 &  0.72\\
& Senegal  & 0.76 &  0.78 &  0.80\\
& Madagascar  & 0.78 &  0.79 &  0.78\vspace{1mm}\\
 Spatial  & Indonesia & 0.71 &  0.72 &  {\bf 0.51}\\
& Senegal  & 0.78 &  0.88 &  0.71\\
& Madagascar  & {\bf 0.67} &  0.70 &  0.72\\


\end{tabular}
\begin{flushleft}
The proportion of held out data points that fall within their 80\% credible intervals. 
Cases where this is below 0.7 are highlighted in bold.
\end{flushleft}
\label{table3}
\end{adjustwidth}
\end{table}




% Coverage results

All models seem to be fairly well calibrated (Table~\ref{table3}).
The proportion of out-of-sample incidence datapoints being within their 80\% credible intervals ranged between 0.51 and 0.88.
However, in most cases coverage was between 0.7 and 0.8 implying the models were a little overconfident in their predictions.


% Overall

Overall, including the spatial information from prevalence surveys yields predictions that are as good or better than the baseline model in all six experiments (three countries and two cross-validation schemes).
The prevalence GP model was as good or better than baseline in five out of six experiments.
In contrast, the joint model was only better than baseline in three out of six experiments.




%figure 3 and 4. data and predicted incidence maps. Indonesia and Senegal only. Fig 3 ind, fig 4 rand. Data, Rand, Spatial for best model? Joint model?

%\begin{figure}
% to be removed before submission
%\includegraphics[width = 0.7\textwidth]{figures/idn_both_cv12_preds.png}
%\caption{{\bf Input incidence data and predicted incidence maps. } 
%The incidence (log10) data (top), predicted log10 incidence from the prevalence GP model for spatially sampled out-of-sample polygons (middle) and predicted log10 %incidence from the joint model for spatially sampled out-of-sample polygons (bottom) for Indonesia. The values plotted for areas with no polygon data are the means across all cross-validation folds.
%}
%\label{predobsmapidn}
%\end{figure}




% Results and Discussion can be combined.
%%%%%%%%%%%%%%%%%%%%%%%%%%%%%%%%%%%%%%%%%%%%%%%%%%%%%%%%%%%%%%%%%%%%%%%%%%%%%%%%%%%%%%%%%%%%%%%%%%%%%
\section*{Discussion}
%%%%%%%%%%%%%%%%%%%%%%%%%%%%%%%%%%%%%%%%%%%%%%%%%%%%%%%%%%%%%%%%%%%%%%%%%%%%%%%%%%%%%%%%%%%%%%%%%%%%%


%Summarise results

We have compared the predictive performance of a polygon-only model and a model that jointly learns from polygon incidence data and prevalence point-surveys.
Overal the prevalence GP model appears to peform best.
While the joint model sometimes performs best it also performs worse than baseline half of the time.
Therefore, fitting a spatial Gaussian process to prevalence points and including these predictions seems to be a more reliable way of using spatial information from prevalence points.


% simple model best
% less fiddly, but none of these models are in packages
%



% joint model benefits
% robust
% lower sd of posteriors
% learn relationship
% disappointed that we couldn't get it to out perform

A full joint model using both prevalence surveys and incidence data is gaining quite a large number of additional degrees of freedom compared to the baseline model and compared to the model that creates a new covariate from prevalence surveys.
Therefore it is worth considering why the performance of this model was generally less good than the simpler prevalence GP model that did not benefit from the additional degrees of freedom.
There are two main reasons why a model fitted jointly on prevalence point-surveys and polygon incidence data may perform worse than a model without that does not prevalence surveys as response data.
Firstly, the data are on different scales.
Here we have used a previously fitted model \cite{cameron2015defining} in order to be able to fit the joint model.
However, this model is imperfect in that it is calibrated with a relatively small amount of matched prevalence and incidence data.
Although we have added the parameter $\alpha$ that scales this relationship, it is a very simple scaling.
Furthermore the true relationship between prevalence and incidence is likely to vary spatially as aspects such as immunity, seasonality and age-structure are not constant \cite{cameron2015defining, battle2015defining, reiner2015seasonality}.
In using a joint model we are accepting these failings in the hope that the benefits of including additional data are stronger than the costs of using mismatched data.
The second aspect of prevalence point-survey data that could reduce the predictive ability of a polygon-only model by adding point data is if the point data is of low quality.
In particular, non-random sampling of individuals can produce biased estimates of prevalence and this will decrease model performance.


For a model to be robust to shortcomings in the prevalence-incidence relationship from \cite{cameron2015defining}, future models could be more flexible in the way they use this relationship.
This could be by estimating the parameters of the polynomial jointly with the rest of the model.
Informative priors based on the original model should be used to regularise this joint fit both to prevent unbelievable inferences but also because if the relationship is too flexible, the information from the prevalence data might not inform the predictions of incidence.
This is particularly true for even more flexible model forms that could be used such as a spline or a Gaussian process on the relationship between prevalence and incidence.

For the model to handle noisy or biased prevalence point-surveys, the modeller can control the iid random effect on the point-surveys, $w_i$ and the prevalence-only intercept $\beta_p$. 
Here we have tried to maximise the influence of the prevalence data by setting the prior based on the belief that the random effect should only explain extra-binomial variation that is impossible to explain with the covariates (i.e. based on the differences in prevalence surveys within the same pixel).
Weakening this prior will allow the iid effect to explain more of the prevalence point-survey variation which both reduces the potential statistical power gained by adding the point-surveys but also reduces the effects of biased or noisy estimates.


Here we have used purely linear covariates.
It has been clearly established that simple linear combinations of environmental covariates cannot fully explain malaria risk \cite{bhatt2017improved}.
A number of methods could be used to include non-linear effects of covariates and interactions into the model.
Firstly, as in \cite{bhatt2017improved}, machine learning models could be fitted to the prevalence data and then predictions from these models could be used as covariates in the full model \cite{lucas2019model}.
This approach is feasible but does not allow any information from the polygons to inform non-linear relationships.
Directly modelling non-linear effects in the full model could be achieved by including simple non-linear functions such as splines \cite{sissoko2017temporal, sewe2017using, hundessa2018projecting}, though the increased model complexity would require more data than was used in Senegal and Madagascar in this study.
Finally, Gaussian process regression, with smoothly varying effects in environmental and geographic space could be used \cite{law2018variational}.
These are computational expensive without variational Bayes or other approximations \cite{law2018variational, ton2018spatial} which can be difficult to derive.
Again these models need a lot of data and careful regularisation for good predictive performance.





%%%%%%%%%%%%%%%%%%%%%%%%%%%%%%%%%%%%%%%%%%%%%%%%%%%%%%%%%%%%%%%%%%%%%%%%%%%%%%%%%%%%%%%%%%%%%%%%%%%%%
\section*{Conclusion}
%%%%%%%%%%%%%%%%%%%%%%%%%%%%%%%%%%%%%%%%%%%%%%%%%%%%%%%%%%%%%%%%%%%%%%%%%%%%%%%%%%%%%%%%%%%%%%%%%%%%%


Overall, we have shown that including spatial information from prevalence surveys generally improves the predictive performance of disaggregation regression of aggregated incidence data.
However, we found that the more complex joint model was unreliable in its predictive performance.
In contrast, summarising the spatial information from the prevalence surveys by fitting a spatial Gaussian process model and using predictions from this model as a new covariate nearly always improved predictive performance.
As disaggregation regression becomes more widely used \cite{weiss2019mapping, battle2019mapping}, methods such as these to improve their performance are particularly needed


\section*{Supporting information}

% Include only the SI item label in the paragraph heading. Use the \nameref{label} command to cite SI items in the text.
%\paragraph*{S1 Fig.}
%\label{S1_Fig}
%{\bf Bold the title sentence.} Add descriptive text after the title of the item (optional).


\section*{Acknowledgments}
Thanks everyone.


\nolinenumbers

% Either type in your references using
% \begin{thebibliography}{}
% \bibitem{}
% Text
% \end{thebibliography}
%
% or
%
% Compile your BiBTeX database using our plos2015.bst
% style file and paste the contents of your .bbl file
% here. See http://journals.plos.org/plosone/s/latex for 
% step-by-step instructions.
% 
\bibliography{Malaria} 

%\begin{thebibliography}{10}


%\end{thebibliography}



\end{document}

